\chapter{User interface}

Detailed worked examples can be found in the source directory under
\lstinline[language=bash]+tests/src/test_adapt_2d.cpp+ and
\lstinline[language=bash]+tests/src/test_adapt_3d.cpp+. Here we
review the key steps required when applying mesh adaptivity to your
model.

\section{Header files}
To be able to perform all the adaptive operations the following
includes are required:
\begin{lstlisting}[language=C++]
#include "Mesh.h"
#include "Surface.h"

#include "MetricField.h"

#include "Coarsen.h"
#include "Refine.h"
#include "Smooth.h"
#include "Swapping.h"
\end{lstlisting}

If you want to make use the VTK I/O interface, typically for the
purpose of unit testing and generating pretty pictures, then also
include:
\begin{lstlisting}[language=C++]
#include "VTKTools.h"
\end{lstlisting}

\section{Initialise mesh}
The first step is to import the mesh into \pragmatic. In the test case
this is done simply be importing a mesh from a VTK XML unstructured
grid file:
\begin{lstlisting}[language=C++]
  Mesh<double, int> *mesh=
    VTKTools<double, int>::import_vtu("../data/smooth_2d.vtu");
\end{lstlisting}
A range of different constructors are available for
\lstinline[language=c++]+Mesh<double, int>+ to handle different kinds of input, such as 2D/3D and distributed meshes. 

For more details see the API documentation at
\lstinline[language=html]+file:///usr/share/doc/pragmatic/html/classMesh.html+

\section{Initialise surface}
The surface mesh must be specified as to preserve geometric
properties. \pragmatic\ does allow on to derive this from the mesh
directly by passing in a reference to the mesh to the class
constructor:
\begin{lstlisting}[language=C++]
  Surface<double, int> *surface = Surface<double, int>(*mesh);
\end{lstlisting}
However, this is not robust in general because floating point
comparisons must be performed to determine if adjacent facets are
co-planar and this can lead to mis-characterisation of gently varying
surfaces. A better approach is to use the method
\lstinline[language=c++]+append_facet+ to explicitly add surface
facets along with their boundary id's.

For more details see the API documentation at
\lstinline[language=html]+file:///usr/share/doc/pragmatic/html/classSurface.html+
 
\section{Initialise metric tensor field}
The metric tensor field, which specifies anisotropically the local
size of elements takes the mesh and the surface as inputs to the
constructor:
\begin{lstlisting}[language=C++]
  MetricField<double, int> metric_field(*mesh, *surface);
\end{lstlisting}
By far the ideal scenario is that the application code supplies the
metric tensor field. This is because in general one must use knowledge
of the problem being solved, and how it has been discretised, to
properly define the necessary errors norms. In order to specify the
metric tensor field use the member function
\lstinline[language=c++]+set_metric+.

A black box implementation of a metric field calculation is also
available. For this you just have to provide a scalar array, a target
error tolerance, and the error norm, $p$:
\begin{lstlisting}[language=C++]
  metric_field.add_field(&(psi[0]), eta, 1);
\end{lstlisting}
If this is called multiple times then the metric fields are
superimposed. Options are available for specifying various constraints
such as the minimum edge length in the mesh, or the maximum number of
elements permitted in the new mesh.

Once you have finished editing the metric field it should be copied to
the mesh, \eg:
\begin{lstlisting}[language=C++]
  metric_field.update_mesh();
\end{lstlisting}

For more details see the API documentation at
\lstinline[language=html]+file:///usr/share/doc/pragmatic/html/classMetricField.html+

\section{Adaptive block}
The main adaptive loop given here is based on an algorithm published
by Li et al. \citep{li20053d}.

\begin{lstlisting}[language=C++]
  // Set upper and lower tolerances on edge length as measured in metric space.
  double L_up = sqrt(2.0);
  double L_low = L_up/2;

  // Initialise adaptive modules
  Coarsen<double, int> coarsen(*mesh, *surface);  
  Smooth<double, int> smooth(*mesh, *surface);
  Refine<double, int> refine(*mesh, *surface);
  Swapping<double, int> swapping(*mesh, *surface);
 
  // Apply initial coarsening
  coarsen.coarsen(L_low, L_up);

  // Initialise the maximum edge length.
  double L_max = mesh->maximal_edge_length();

  double alpha = sqrt(2.0)/2;  
  for(size_t i=0;i<10;i++){
    // Used to throttle the coarsening and refinement.
    double L_ref = std::max(alpha*L_max, L_up);
    
    // Refine mesh.
    refine.refine(L_ref);

    // Coarsen mesh.
    coarsen.coarsen(L_low, L_ref);

    // Improve quality through swapping
    swapping.swap(0.95);

    // Update the maximum edge length and check convergence of algorithm,
    L_max = mesh->maximal_edge_length();
    if((L_max-L_up)<0.01)
      break;
  }

  // Defragment the memory usage.  
  std::map<int, int> active_vertex_map;
  mesh->defragment(&active_vertex_map);
  surface.defragment(&active_vertex_map);

  // Apply vertex smoothing.
  smooth.smooth("optimisation Linf", 200);
\end{lstlisting}

\section{Diagnostics}
A number of diagnostics are available to evaluate the quality of the
mesh before, during and after adapting the mesh. To get a general
summery printed to \lstinline[language=bash]+stdout+ use:
\begin{lstlisting}[language=C++]
  mesh->verify();
\end{lstlisting}

Alternatively, the values can be retrieved within the code, \eg:
\begin{lstlisting}[language=C++]
  qmean = mesh->get_qmean();
  qrms = mesh->get_qrms();
  qmin = mesh->get_qmin();
\end{lstlisting}

If you have enabled \lstinline[language=bash]+VTK+ then you can also
dump results to \lstinline[language=bash]+VTK+ files for visualisation
purposes, \eg:
\begin{lstlisting}[language=C++]
  VTKTools<double, int>::export_vtu("../data/test_adapt_2d", mesh, &(psi[0]));
  VTKTools<double, int>::export_vtu("../data/test_adapt_2d_surface", &surface);
\end{lstlisting}

\section{Exporting the mesh}
The classes \lstinline[language=c++]+Mesh+ and
\lstinline[language=c++]+Surface+ provide member functions to copy
retrieve the adapted mesh from \pragmatic. These methods are clearly
identified by the prefix \lstinline[language=c++]+get_+, see
\lstinline[language=html]+file:///usr/share/doc/pragmatic/html/classMesh.html+
and
\lstinline[language=html]+file:///usr/share/doc/pragmatic/html/classSurface.html+
for details.
